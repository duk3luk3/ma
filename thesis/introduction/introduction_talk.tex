\documentclass[a4paper]{article}
\usepackage[utf8]{inputenc}
\usepackage{amsmath}
\usepackage{fancyhdr, graphicx}
\usepackage{hyperref}
\usepackage[margin=3cm]{geometry}

\renewcommand{\headrulewidth}{0pt}

\fancyhead[L]{
\includegraphics[width=5cm]{i8_logo.pdf}
}
\fancyhead[R]{
\includegraphics[width=8cm]{tum_logo.pdf}
}

\title{Temporal Behavior of PC-based Packet Processing}
\date{\today}
\author{Lukas Erlacher}

\begin{document}

\maketitle
\thispagestyle{fancy}

\begin{flushleft}
Supervisor: Prof. Dr.-Ing. Georg Carle\\
Advisor: Paul Emmerich (M.Sc.), Daniel Raumer (M.Sc.), Florian Wohlfart (M.Sc.)\\
Type of work: Master's thesis
\end{flushleft}


\section*{Introduction}
In the context of the MEMPHIS project we analyse the packet processing performance in the linux kernel with respect to latency. We define various usecases such as packet forwarding with Open vSwitch and measure them in the MEMPHIS testbed.


\section*{Motivation}
The Linux Kernel is a pervasive part of the peripheral routing infrastructure of the internet, especially due to increasing virtualization of servers. Open vSwitch is an implementation of Software Defined Networking that uses packet forwarding in the Linux Kernel as main part of its data plane. The most important usecase of Open vSwitch is the routing of virtual networks between the hypervisor and VMs in virtual server farms.

One of the main performance metrics for this is end-to-end latency. On the software router, latency is created by processing and copying packets between layers in the NIC, NIC driver, NAPI kernel implementation, and the Open vSwitch data plane or user space applications. There is a large number of parameters that can be adjusted such as buffer sizes, interrupt throttling rates, or polling intervals.

Related work concerning software routers is not comprehensive and does not encompass new features in the Linux Kernel such as low latency network device polling. Existing results and models must be updated to account for these new features.

\section*{Approach}
Initially, a basic model for the latency sources in a simple software router scenario will be validated. It will be based on existing MEMPHIS publications such as the Bachelor Thesis ''A Model for Performance Prediction in PC-based Packet Processing Systems`` (Scholz) and ''A Study of Networking Software Induced Latency`` (Beifuß et.al.).

We will then measure the influence of varying parameters such as various buffer sizes, the NIC's Interrupt Throttle Rate, and NAPI's Low Latency Network Device Polling and update the model accordingly.

This updated model will be validated with additional usecases.

\vspace{1em}
Additional sources:
\begin{itemize}
  \item Low-latency Ethernet device polling, \url{http://lwn.net/Articles/551284/}
  \item Intel X450 Data Sheet (Interrupt Throttling)
  \item Bolla/Bruschi, Linux Software Router: Data Plane Optimization and Performance Evaluation
  \item Larsen et.al., Architectural Breakdown of End-to-End Latency in a TCP/IP Network
\end{itemize}


\section*{Schedule}

\begin{tabular}{ll}
  November 2014 & Literature research and first tests (testbed familiarization)\\
  December 2014 & Tests with parameters, set up basic model and structure of thesis\\
  January 2015  & Run tests\\
  February 2015 & Refine tests and model, pick scenarios to use in thesis\\
  March 2015    & Intermediary presentation with first results\\
  April 2015    & Finish tests and analyses\\
  May 2015      & Finish and submit thesis\\
\end{tabular}
\end{document}
